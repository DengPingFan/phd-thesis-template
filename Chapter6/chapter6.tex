\chapter{Conclusion}
\label{chapter6}

%\setlength{\epigraphrule}{0pt}
\epigraph{\textit{If you can't fly then run, if you can't run then walk, if you can't walk then crawl, but whatever you do you have to keep moving forward.}}{ -- Martin Luther King Jr.}

% **************************** Define Graphics Path **************************
\ifpdf
    \graphicspath{{Chapter6/Figs/Raster/}{Chapter6/Figs/PDF/}{Chapter6/Figs/}}
\else
    \graphicspath{{Chapter6/Figs/Vector/}{Chapter6/Figs/}}
\fi

\section{Summary}

We made following contributions:

\begin{itemize}
	\item We propose using a segment-based approach to overcome the limitations of  the video-based approaches. The basic idea is to examine shorter segments instead of using the representative frames or entire video. We carry thorough experiments to verify our proposed method by investigating different strategies to decompose a video into segments. These strategies include uniform segment sampling and 
	
	\item We propose a new video pooling strategy, called sum-max video pooling, to deal with noisy information in complex videos. This pooling technique is based on the layer structure of video. Basically, we apply sum pooling at the low layer representation while using max pooling at the high layer representation. Sum pooling is used to keep sufficient relevant features at the low layer, while max pooling is used to retrieve the most relevant features at the high layer, therefore it can discard irrelevant features in the final video representation. 
	
	\item We propose a new method, named Event-driven Multiple Instance Learning (EDMIL), to learn key evidences for complex event detection. We treat each segment as an instance and model it in a multiple instance learning framework \cite{andrews2002support}, where each video is a "bag". The instance-event similarity is quantized into different levels of relatedness. Intuitively, the most (ir)relevant instances should have higher (dis)similarities. Therefore, we propose to learn the instance labels by jointly optimizing the instance classifier and its related level.	
	
\end{itemize}
	
\section{Future Work}

